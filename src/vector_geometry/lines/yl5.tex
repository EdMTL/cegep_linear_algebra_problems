%-----------------------------------------------------
% index key words
%-----------------------------------------------------
\index{line, parametric equation of line}
\index{skew line, line!skew}

%-----------------------------------------------------
% name, leave blank
% title, if the exercise has a name i.e. Hilbert's matrix
% difficulty = n, where n is the number of stars
% origin = "\cite{ref}"
%-----------------------------------------------------
\begin{Exercise}[
name={},
title={}, 
difficulty=0,
origin={\cite{YL}}]
Given the following lines which are all skew to each other:
\[
\begin{array}{lllllllll}
\mathcal{L}_1 & : & \; (x, y, z)=(1, 0, 0) & + & t_1(1, 2, 0)\\
\mathcal{L}_2 & : & \; (x, y, z)=(1, 1, 0) & + & t_2(1, 0, 1)\\
\mathcal{L}_3 & : & \; (x, y, z)=(1, 0, 1) & + & t_3(1, 2, 3)
\end{array}
\]
here $t_1,\;t_2,\;t_3\in\mathbb{R}$. Consider a line $\mathcal{L}_4$ that is parallel
to $\mathcal{L}_3$ and intersects both $\mathcal{L}_1$ and $\mathcal{L}_2$.  Find the
points of intersection of $\mathcal{L}_4$ with $\mathcal{L}_1$ and $\mathcal{L}_4$ with
$\mathcal{L}_2$.
\end{Exercise}
\begin{Answer}
$\left(\frac{4}{3},\;\frac{2}{3},\;0\right)$, $\left(\frac{3}{2},\;1,\;\frac{1}{2}\right)$
\end{Answer}
