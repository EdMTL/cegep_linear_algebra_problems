%-----------------------------------------------------
% index key words
%-----------------------------------------------------
\index{line, parametric equation of line}
\index{skew line, line!skew}
\index{distance}
\index{distance!skew lines}

%-----------------------------------------------------
% name, leave blank
% title, if the exercise has a name i.e. Hilbert's matrix
% difficulty = n, where n is the number of stars
% origin = "\cite{ref}"
%-----------------------------------------------------
\begin{Exercise}[
name={},
title={}, 
difficulty=0,
origin={\cite{YL}}]
Given 
\[
\begin{array}{lllllllll}
\mathcal{L}_1 & : & \; (x, y, z)=(1, 0, 1) & + & t_1 (-2,-1,0)\\
\mathcal{L}_2 & : & \; (x, y, z)=(-2, -1, 2) & + & t_2 (1,0,1) 
\end{array}
\]
where $t_1,\;t_2\in\Re$.
\Question Determine whether the two lines intersect, are parallel or are skew lines.
\Question Find the shortest distance between the two lines.
\Question Find the equation of a line which is orthogonal to the direction of both given lines and passes through $\mathcal{L}_1$.  Is the line unique?
\end{Exercise}
\begin{Answer}
\Question Skew lines.
\Question $\sqrt{\frac23}$
\Question The line is not unique, $(x,y,z)=(1,0,1)+t(-1,2,1)$ where $t\in\Re$.
\end{Answer}
