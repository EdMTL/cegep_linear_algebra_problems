%-----------------------------------------------------
% index key words
%-----------------------------------------------------
\index{matrix power}

%-----------------------------------------------------
% name, leave blank
% title, if the exercise has a name i.e. Hilbert's matrix
% difficulty = n, where n is the number of stars
% origin = "\cite{ref}"
%-----------------------------------------------------
\begin{Exercise}[
name={},
title={}, 
difficulty=0,
origin={\cite{BS}}]
Determine whether the following statements are true or false for any $n\times n$
matrices $A$ and $B$.  If the statement is false provide a counterexample. If the
statement is true provide a proof of the statement.
\Question If $A^2=0$ then $A=0$.
\Question $A$ is said to be \emph{skew-symmetric} if $A^T=-A$.  If $A$ is symmetric and skew-symmetric then $A=0$.
\Question If $AA^T=A$ then $A$ is symmetric.
\Question If $A^2$ is an elementary matrix then $A$ is an elementary matrix.
\end{Exercise}

\begin{Answer}
\Question False, if $A=\begin{mat} 0 & 0\\ 1 & 0\end{mat}$ then $A^2=0$ but $A\neq0$. 
\Question True, by the premises $A^T=A$ and $A^T=-A$, it follows that
\begin{eqnarray*}
A^T&=&A^T\\
A &=&-A\\
2A &=& 0\\
A & = &0
\end{eqnarray*}
\Question True, since $A^T=(AA^T)^T=(A^T)^TA^T=AA^T=A$.
\Question False, if $A=\begin{mat}-1 & 0\\ 0 & -1\end{mat}$ then $A^2=I$ which is an elementary matrix but $A$ is not an elementary matrix.
\end{Answer}
