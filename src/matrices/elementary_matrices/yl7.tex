%-----------------------------------------------------
% index key words
%-----------------------------------------------------
\index{matrix!elementary}
\index{matrix!inverse}
\index{matrix!row equivalent}
\index{matrix!Hilbert's}

%-----------------------------------------------------
% name, leave blank
% title, if the exercise has a name i.e. Hilbert's matrix
% difficulty = n, where n is the number of stars
% origin = "\cite{ref}"
%-----------------------------------------------------
\begin{Exercise}[
name={},
title={}, 
difficulty=0,
origin={\cite{YL}}]
Given the following $5\times 5$ matrices: $A$, \emph{Hilbert's matrix} $H$ and its inverse
\[
A=
\begin{bmatrix}
\frac{1}{5} & \frac{1}{6} & \frac{1}{7} & \frac{1}{8} & \frac{1}{9}\vspace{0.1em}\\
1 & \frac{2}{3} & \frac{1}{2} & \frac{2}{5} & \frac{1}{3} \vspace{0.1em}\\
\frac{1}{3} & \frac{1}{4} & \frac{1}{5} & \frac{1}{6} & \frac{1}{7} \vspace{0.1em}\\
\frac{1}{4} & \frac{1}{5} & \frac{1}{6} & \frac{1}{7} & \frac{1}{8} \vspace{0.1em}\\
\frac{1}{1} & \frac{1}{2} & \frac{1}{3} & \frac{1}{4} & \frac{1}{5} 
\end{bmatrix},\;\;
H=
\begin{bmatrix}
  \frac{1}{1} & \frac{1}{2} & \frac{1}{3} & \frac{1}{4} & \frac{1}{5} \vspace{0.1em}\\
  \frac{1}{2} & \frac{1}{3} & \frac{1}{4} & \frac{1}{5} & \frac{1}{6} \vspace{0.1em}\\
  \frac{1}{3} & \frac{1}{4} & \frac{1}{5} & \frac{1}{6} & \frac{1}{7} \vspace{0.1em}\\
  \frac{1}{4} & \frac{1}{5} & \frac{1}{6} & \frac{1}{7} & \frac{1}{8} \vspace{0.1em}\\
  \frac{1}{5} & \frac{1}{6} & \frac{1}{7} & \frac{1}{8} & \frac{1}{9}
\end{bmatrix},\]
\[
H^{-1}=
\begin{bmatrix}
         25    &    -300    &    1050    &   -1400    &     630  \\
        -300   &     4800   &   -18900   &    26880   &   -12600 \\
        1050   &   -18900   &    79380   &  -117600   &    56700 \\
       -1400   &    26880   &  -117600   &   179200   &   -88200 \\
         630   &   -12600   &    56700   &   -88200   &    44100 
\end{bmatrix}\]
\Question Show that $H$ is row equivalent to $A$ by finding two elementary matrices $E_i$ such that $E_2E_1H=A$.
\Question Find the inverse of $A$.
\end{Exercise}

\begin{Answer}
\Question $A$ can be obtained from $H$ by performing $R_1\leftrightarrow R_5$ and $2R_2\to R_2$.  Hence
\[
E_1=
\begin{mat}
0 & 0 & 0 & 0 & 1\\
0 & 1 & 0 & 0 & 0\\
0 & 0 & 1 & 0 & 0\\
0 & 0 & 0 & 1 & 0\\
1 & 0 & 0 & 0 & 0
\end{mat},\;
E_2=
\begin{mat}
1 & 0 & 0 & 0 & 0\\
0 & 2 & 0 & 0 & 0\\
0 & 0 & 1 & 0 & 0\\
0 & 0 & 0 & 1 & 0\\
0 & 0 & 0 & 0 & 1
\end{mat}\]
\Question $A^{-1} = H^{-1}E^{-1}_1E^{-1}_2\\
H^{-1}
E_1
\begin{mat}
1 & 0 & 0 & 0 & 0\\
0 & \frac12 & 0 & 0 & 0\\
0 & 0 & 1 & 0 & 0\\
0 & 0 & 0 & 1 & 0\\
0 & 0 & 0 & 0 & 1
\end{mat}\\
=
\begin{mat}
630 & -150 & 1050 & -1400 & 25\\
-12600 & 2400 & -1890 & 26880 & -300\\
56700 & -9450 & 79380 & -117600 & 1050\\
-88200 & 13440 & -117600 & 179200 & -1400\\
44100 & -6300 & 56700 & -88200 & 630
\end{mat}
$
\end{Answer}
